\documentclass[10pt,letterpaper]{article}
\begin{document}
% Include only the SI item label in the subsection heading. Use the \nameref{label} command to cite SI items in the text.

%\section*{S1 Appendix}\label{S1_Appendix}
%{\bf Creativity Corpus.}  The following 30 papers were used as the {\em creativity corpus} for this work:
%
%\begin{itemize}
%\item T. M. Amabile. The social psychology of creativity: A componential conceptualization. Journal of Personality and Social Psychology, 45(2):357-376, 1983.
%\item M. A. Boden. Precis of The Creative Mind: Myths and mechanisms. Behavioural and Brain Sciences, 17(3):519-570, 1994.
%\item D. T. Campbell. Blind variation and selective retentions in creative thought as in other knowledge processes. Psychological Review, 67(7):380-400, 1960.
%\item S. Colton, A. Pease, and G. Ritchie. The effect of input knowledge on creativity. In Proceedings of Workshop Program of ICCBR-Creative Systems: Approaches to Creativity in AI and Cognitive Science, 2001.
%\item M. Csikszentmihalyi. Motivation and creativity: Toward a synthesis of structural and energistic approaches to cognition. New Ideas in Psychology, 6(2):159-176, 1988.
%\item M. Dellas and E. L. Gaier. Identification of creativity: The individual. Psychological Bulletin, 73(1):55- 73, 1970.
%\item A. Dietrich. The cognitive neuroscience of creativity. Psychonomic Bulletin \& Review, 11(6):1011-1026, 2004.
%\item G. Domino. Identification of potentially creative persons from the adjective check list. Journal of Consulting and Clinical Psychology, 35(1):48-51, 1970.
%\item W. Duch. Intuition, insight, imagination and creativity. IEEE Computational Intelligence Magazine, 2(3):40-52, 2007.
%\item C. S. Findlay and C. J. Lumsden. The creative mind: Toward an evolutionary theory of discovery and innovation. Journal of Social and Biological Systems, 11(1):3-55, 1988.
%\item C. M. Ford. A theory of individual creative action in multiple social domains. The Academy of Management Review, 21(4):1112-1142, 1996.
%\item J. Gero. Creativity, emergence and evolution in design. Knowledge-Based Systems, 9(7):435-448, 1996. 
%\item H. G. Gough. A creative personality scale for the adjective checklist. Journal of Personality and Social Psychology, 37(8):1398-1405, 1979.
%\item J. P. Guilford. Creativity. American Psychologist, 5:444-454, 1950.
%\item Z. Ivcevic. Creativity map: Toward the next generation of theories of creativity. Psychology of Aesthetics, Creativity, and the Arts, 3(1):17-21, 2009.
%\item K. H. Kim. Can we trust creativity tests? A review of the Torrance tests of creative thinking (TTCT). Creativity Research Journal, 18(1):3-14, 2006.
%\item L. A. King, L. McKee Walker, and S. J. Broyles. Creativity and the five-factor model. Journal of Research in Personality, 30(2):189-203, 1996.
%\item R. R. McCrae. Creativity, divergent thinking, and openness to experience. Journal of Personality and Social Psychology, 52(6):1258-1265, 1987.
%\item S. A. Mednick. The associative basis of the creative process. Psychological Review, 69(3):220-232, 1962. 
%\item M. D. Mumford and S. B. Gustafson. Creativity syndrome: Integration, application, and innovation. Psychological Bulletin, 103(1):27-43, 1988. 
%\item M. T. Pearce, D. Meredith, and G. A. Wiggins. Motivations and methodologies for automation of the compositional process. Musicae Scientae, 6(2):119-147, 2002.
%\item J. A. Plucker, R. A. Beghetto, and G. T. Dow. Why isn't creativity more important to educational psychologists? Potentials, pitfalls, and future directions in creativity research. Educational Psychologist, 39(2):83-96, 2004.
%\item R. Richards, D. K. Kinney, M. Benet, and A. P. C. Merzel. Assessing everyday creativity: Characteristics of the lifetime creativity scales and validation with three large samples. Journal of Personality and Social Psychology, 54(3):476-485, 1988.
%\item G. Ritchie. The transformational creativity hypothesis. New Generation Computing, 24(3):241-266, 2006.
%\item G. Ritchie. Some empirical criteria for attributing creativity to a computer program. Minds and Machines, 17:67-99, 2007.
%\item D. L. Rubenson and M. A. Runco. The psychoeconomic approach to creativity. New Ideas in Psychology, 10(2):131-147, 1992.
%\item M. A. Runco and I. Chand. Cognition and creativity. Educational Psychology Review, 7(3):243-267, 1995.
%\item D. K. Simonton. Creativity: Cognitive, personal, developmental, and social aspects. American Psychologist, 55(1):151-158, 2000.
%\item J. R. Suler. Primary process thinking and creativity. Psychological Bulletin, 88(1):144-165, 1980.
%\item G. A. Wiggins. A preliminary framework for description, analysis and comparison of creative systems. Knowledge-Based Systems, 19(7):449-458, 2006.
%\end{itemize}
%
%

%
\section*{S2 Appendix}\label{S2_Appendix}
{\bf Non-Creativity Corpus.} The following 60 papers were used as the {\em non-creativity corpus} for this work:

\begin{itemize}
\item C. Ames and J. Archer. Achievement goals in the classroom: Students' learning strategies and motivation processes. Journal of Educational Psychology, 80(3):260-267, 1988. % Cited by:  (since 1996) 639.
\item J. Anderson and D. Gerbing. Structural equation modeling in practice: A review and recommended two-step approach. Psychological Bulletin, 103(3):411-423, 1988. % Cited by:  (since 1996) 3884.
\item J. Arnett. Emerging adulthood: A theory of development from the late teens through the twenties. American Psychologist, 55(5):469-480, 2000. % Cited by:  (since 1996) 800.
\item M. Arulampalam, S. Maskell, N. Gordon, and T. Clapp. A tutorial on particle filters for online nonlinear/non-gaussian bayesian tracking. IEEE Transactions on Signal Processing, 50(2):174-188, 2002. % Cited by:  (since 1996) 1984.
\item A. Baddeley. Exploring the central executive. Quarterly Journal of Experimental Psychology Section A: Human Experimental Psychology, 49(1):5-28, 1996. % Cited by:  (since 1996) 672.
\item T. Baker, M. Piper, D. McCarthy, M. Majeskie, and M. Fiore. Addiction motivation reformulated: An affective processing model of negative reinforcement. Psychological Review, 111(1):33-51, 2004. % Cited by:  (since 1996) 177.
\item P. Barnett and I. Gotlib. Psychosocial functioning and depression: Distinguishing among antecedents, concomitants, and consequences. Psychological Bulletin, 104(1):97-126, 1988. % Cited by:  (since 1996) 366.
\item J. Baron. Nonconsequentialist decisions. Behavioral and Brain Sciences, 17(1):1-42, 1994. % Cited by:  (since 1996) 68.
\item F. Beach. The snark was a boojum. American Psychologist, 5(4):115-124, 1950. % Cited by:  (since 1996) 32.
\item M. Belkin, P. Niyogi, and V. Sindhwani. Manifold regularization: A geometric framework for learning from labeled and unlabeled examples. Journal of Machine Learning Research, 7:2399-2434, 2006. % Cited by:  (since 1996) 145.
\item G. Bonanno. Loss, trauma, and human resilience: Have we underestimated the human capacity to thrive after extremely aversive events? American Psychologist, 59(1):20-28, 2004. % Cited by:  (since 1996) 352.
\item T. Chan and L. Vese. Active contours without edges. IEEE Transactions on Image Processing, 10(2):266-277, 2001. % Cited by:  (since 1996) 1520.
\item H. Cheng and J. Schweitzer. Cultural values reflected in Chinese and U.S. television commercials. Journal of Advertising Research, 36(3):27-45, 1996. % Cited by:  (since 1996) 88.
\item C. Coello Coello. Evolutionary multi-objective optimization: A historical view of the field. IEEE Computational Intelligence Magazine, 1(1):28-36, 2006. % Cited by:  (since 1996) 109.
\item D. Comaniciu and P. Meer. Mean shift: A robust approach toward feature space analysis. IEEE Transactions on Pattern Analysis and Machine Intelligence, 24(5):603-619, 2002.  % Cited by:  (since 1996) 1613
\item L. Cronbach and L. Furby. How we should measure `change': Or should we? Psychological Bulletin, 74(1):68-80, 1970. % Cited by: (since 1996) 526.
\item M. Davis. Measuring individual differences in empathy: Evidence for a multidimensional approach. Journal of Personality and Social Psychology, 44(1):113-126, 1983. % Cited by:  (since 1996) 631.
\item J. Dem\~{s}ar. Statistical comparisons of classifiers over multiple data sets. Journal of Machine Learning Research, 7:1-30, 2006. % Cited by:  (since 1996) 389.
\item M. Dorigo, M. Birattari, and T. St\"{u}tzle. Ant colony optimization artificial ants as a computational intelligence technique. IEEE Computational Intelligence Magazine, 1(4):28-39, 2006. % Cited by:  (since 1996) 134.
\item E. Fischer and J. Turner. Orientations to seeking professional help: Development and research utility of an attitude scale. Journal of Consulting and Clinical Psychology, 35(1 PART 1):79-90, 1970. % Cited by:  (since 1996) 189.
\item J. Gibson. Observations on active touch. Psychological Review, 69(6):477-491, 1962. % Cited by:  (since 1996) 179.
\item A. Gopnik and J. Astington. Children's understanding of representational change and its relation to the understanding of false belief and the appearance-reality distinction. Child development, 59(1):26-37, 1988. % Cited by:  (since 1996) 314.
\item S. Gosling, S. Vazire, S. Srivastava, and O. John. Should we trust web-based studies? A comparative analysis of six preconceptions about internet questionnaires. American Psychologist, 59(2):93-104, 2004. % Cited by:  (since 1996) 313.
\item J. Gray. The psychophysiological basis of introversion-extraversion. Behaviour Research and Therapy, 8(3):249-266, 1970. % Cited by:  (since 1996) 225.
\item B. Grosz and S. Kraus. Collaborative plans for complex group action. Artificial Intelligence, 86(2):269- 357, 1996. % Cited by:  (since 1996) 275.
\item P. Groves and R. Thompson. Habituation: A dual-process theory. Psychological Review, 77(5):419-450, 1970. % Cited by:  (since 1996) 375.
\item M. Hall, J. Anderson, S. Amarasinghe, B. Murphy, S.-W. Liao, E. Bugnion, and M. Lam. Maximizing multiprocessor performance with the SUIF compiler. Computer, 29(12):84-89, 1996. % Cited by:  (since 1996) 161.
\item F. Happ\'{e}. The role of age and verbal ability in the theory of mind task performance of subjects with autism. Child development, 66(3):843-855, 1995. % Cited by:  (since 1996) 285.
\item C. Harland. Supply chain management: Relationships, chains and networks. British Journal of Management, 7(SPEC. ISS.):S63-S80, 1996. % Cited by:  (since 1996) 172.
\item S. Hayes, K. Strosahl, K. Wilson, R. Bissett, J. Pistorello, D. Toarmino, M. Polusny, T. Dykstra, S. Batten, J. Bergan, S. Stewart, M. Zvolensky, G. Eifert, F. Bond, J. Forsyth, M. Karekla, and S. McCurry. Measuring experiential avoidance: A preliminary test of a working model. Psychological Record, 54(4):553-578, 2004. % Cited by:  (since 1996) 168.
\item J. Hirsch and K. L\"{u}cke. Overview no. 76. mechanism of deformation and development of rolling textures in polycrystalline f.c.c. metals-i. description of rolling texture development in homogeneous cuzn alloys. Acta Metallurgica, 36(11):2863-2882, 1988. % Cited by:  (since 1996) 243.
\item P. Killeen. Mathematical principles of reinforcement. Behavioral and Brain Sciences, 17(1):105-172, 1994. % Cited by:  (since 1996) 86.
\item P. Kirschner, J. Sweller, and R. Clark. Why minimal guidance during instruction does not work: An analysis of the failure of constructivist, discovery, problem-based, experiential, and inquiry-based teaching. Educational Psychologist, 41(2):75-86, 2006. % Cited by:  (since 1996) 139.
\item S. Liao. Notes on the homotopy analysis method: Some definitions and theorems. Communications in Nonlinear Science and Numerical Simulation, 14(4):983-997, 2009. % Cited by:  (since 1996) 56.
\item C. Lord, L. Ross, and M. Lepper. Biased assimilation and attitude polarization: The effects of prior theories on subsequently considered evidence. Journal of Personality and Social Psychology, 37(11):2098- 2109, 1979. % Cited by:  (since 1996) 523.
\item S. Luthar, D. Cicchetti, and B. Becker. The construct of resilience: A critical evaluation and guidelines for future work. Child Development, 71(3):543-562, 2000. % Cited by:  (since 1996) 566.
\item G. Mandler. Recognizing: The judgment of previous occurrence. Psychological Review, 87(3):252-271, 1980. % Cited by:  (since 1996) 877.
\item G. Miller and J. Selfridge. Verbal context and the recall of meaningful material. The American journal of psychology, 63(2):176-185, 1950. % Cited by:  (since 1996) 57.
\item S. Miller. Monitoring and blunting: Validation of a questionnaire to assess styles of information seeking under threat. Journal of Personality and Social Psychology, 52(2):345-353, 1987. % Cited by:  (since 1996) 385.
\item G. Navarro and V. M\"{a}kinen. Compressed full-text indexes. ACM Computing Surveys, 39(1), 2007. % Cited by:  (since 1996) 62.
\item M. Nissen and P. Bullemer. Attentional requirements of learning: Evidence from performance measures. Cognitive Psychology, 19(1):1-32, 1987. % Cited by:  (since 1996) 729.
\item J. Payne, J. Bettman, and E. Johnson. Adaptive strategy selection in decision making. Journal of Experimental Psychology: Learning, Memory, and Cognition, 14(3):534-552, 1988. % Cited by:  (since 1996) 300.
\item J. Prochaska, C. DiClemente, and J. Norcross. In search of how people change: Applications to addictive behaviors. American Psychologist, 47(9):1102-1114, 1992. % Cited by:  (since 1996) 2742.
\item T. Richardson, M. Shokrollahi, and R. Urbanke. Design of capacity-approaching irregular low-density parity-check codes. IEEE Transactions on Information Theory, 47(2):619-637, 2001. % Cited by:  (since 1996) 1162.
\item W. Rozeboom. The fallacy of the null-hypothesis significance test. Psychological Bulletin, 57(5):416-428, 1960. % Cited by:  (since 1996) 133.
\item C. Rusbult. A longitudinal test of the investment model: The development (and deterioration) of satisfaction and commitment in heterosexual involvements. Journal of Personality and Social Psychology, 45(1):101-117, 1983. % Cited by:  (since 1996) 301.
\item T. Ryan. Significance tests for multiple comparison of proportion, variance, and other statistics. Psychological Bulletin, 57(4):318-328, 1960. % Cited by:  (since 1996) 154.
\item W. Schultz. Behavioral theories and the neurophysiology of reward. Annual Review of Psychology, 57:87-115, 2006. % Cited by:  (since 1996) 226.
\item E. Sirin, B. Parsia, B. Grau, A. Kalyanpur, and Y. Katz. Pellet: A practical OWL-DL reasoner. Web Semantics, 5(2):51-53, 2007. % Cited by:  (since 1996) 125.
\item T. Srull and R. Wyer. The role of category accessibility in the interpretation of information about persons: Some determinants and implications. Journal of Personality and Social Psychology, 37(10):1660- 1672, 1979. % Cited by:  (since 1996) 374.
\item J. Steiger. Tests for comparing elements of a correlation matrix. Psychological Bulletin, 87(2):245-251, 1980. % Cited by:  (since 1996) 705.
\item L. Steinberg, S. Lamborn, S. Dornbusch, and N. Darling. Impact of parenting practices on adolescent achievement: authoritative parenting, school involvement, and encouragement to succeed. Child development, 63(5):1266-1281, 1992. % Cited by:  (since 1996) 435.
\item D. Tao, X. Li, X. Wu, W. Hu, and S. Maybank. Supervised tensor learning. Knowledge and Information Systems, 13(1):1-42, 2007. % Cited by:  (since 1996) 54.
\item A. Tellegen, D. Lykken, T. Bouchard Jr., K. Wilcox, N. Segal, and S. Rich. Personality similarity in twins reared apart and together. Journal of Personality and Social Psychology, 54(6):1031-1039, 1988. % Cited by:  (since 1996) 398.
\item L. Thomas and D. Ganster. Impact of family-supportive work variables on work-family conflict and strain: A control perspective. Journal of Applied Psychology, 80(1):6-15, 1995. % Cited by:  (since 1996) 339.
\item I. Thompson. Coupled reaction channels calculations in nuclear physics. Computer Physics Reports, 7(4):167-212, 1988. % Cited by:  (since 1996) 360.
\item E. Tulving. Subjective organization in free recall of ``unrelated'' words. Psychological Review, 69(4):344- 354, 1962. % Cited by:  (since 1996) 121.
\item U. Von Luxburg. A tutorial on spectral clustering. Statistics and Computing, 17(4):395-416, 2007. % Cited by:  (since 1996) 107.
\item J. Williams, A. Mathews, and C. MacLeod. The emotional stroop task and psychopathology. Psychological Bulletin, 122(1):3-24, 1996. % Cited by:  (since 1996) 716.
\item J. Wright, A. Yang, A. Ganesh, S. Sastry, and Y. Ma. Robust face recognition via sparse representation. IEEE Transactions on Pattern Analysis and Machine Intelligence, 31(2):210-227, 2009. % Cited by:  (since 1996) 46.
\end{itemize}

\end{document}
